\section{Metodología de investigación aplicada}
Se realizó una revisión sistemática de artículos científicos publicados entre los años 2015 y 2025, seleccionados por su relevancia en el ámbito de la ingeniería de software, la gestión de proyectos y la innovación tecnológica. Las fuentes fueron obtenidas de bases de datos académicas y revistas especializadas, priorizando investigaciones indexadas y de acceso verificable. Entre ellas se encuentran Procedia Computer Science, Information and Software Technology, International Journal of Technology, Management and Humanities, World Journal of Advanced Research and Reviews e IEEE Access, entre otras \cite{Stray2025,Jain2025,PriyankaMalla2025,Solige2025,Qamar2025,ElAouni2025}.

En primer lugar, se realizó la búsqueda, selección y clasificación de los artículos según criterios temáticos: fundamentos del pensamiento ágil, estudios comparativos entre metodologías (Scrum, Kanban, Waterfall), integración con tecnologías emergentes (DevOps, Cloud, IA), gestión del riesgo, diversidad de equipos y seguridad del software \cite{William2025,LS2025,Jibgah2025,Khan2025,Ok2025}. En segundo lugar, se llevó a cabo una lectura crítica y analítica, enfocada en identificar coincidencias, diferencias y aportes relevantes entre los distintos enfoques. Finalmente, se elaboró una síntesis interpretativa que permitió articular los resultados de los estudios bajo una perspectiva integradora y contextualizada \cite{Maidin2025,JawishAlgorithm}.

\begin{figure}[h!]
    \centering
    \includegraphics[width=0.85\linewidth]{graphics/grafica2.png}
    \caption{Proceso de búsqueda, selección y clasificación de los artículos analizados.}
    \label{fig:grafica2}
\end{figure}

La investigación se apoyó en un proceso de reflexión cruzado en el que se contrastaron los hallazgos teóricos con las implicaciones prácticas descritas en los estudios. Esto permitió construir una visión global sobre la evolución de las metodologías ágiles, su impacto en la calidad, la productividad y la cultura de trabajo en entornos de desarrollo de software \cite{PriyankaMalla2025,William2025}.

La metodología se enfoca en el estudio de las investigaciones como herramienta para comprender cómo las prácticas ágiles, más allá de ser marcos técnicos, representan una transformación cultural que une tecnología, liderazgo y colaboración en la creación de soluciones digitales sostenibles \cite{LopezMenendezUnidad,Stray2025,LS2025}.

La lectura de los artículos deja claro que los temas principales son Scrum, Kanban, DevOps, nube, automatización, seguridad, híbridos ágiles, lógica difusa, colaboración, documentación, diversidad en los equipos y mejora continua \cite{Reich2025,ElAouni2025,Jibgah2025,Aguayo2025,Qamar2025}. La intención no fue clasificar los artículos de manera aislada, sino reconocer cómo cada uno aportaba una pieza al panorama general del desarrollo moderno.

Se tomó como eje común la evolución de Agile y su interacción con tecnologías y enfoques actuales. Según el análisis de las investigaciones estás se pueden conectar, por ejemplo: cómo DevOps se relaciona con automatización, cómo la documentación en pares influye en la calidad, cómo Scrum funciona en equipos distribuidos, o cómo Lean Six Sigma y Agile se complementan \cite{Reich2025,Qamar2025,Solige2025,Shriram2025,Jibgah2025}.

Después se organizaron las ideas siguiendo un hilo lógico, primero el origen de Agile y sus fundamentos; luego las adaptaciones según contexto, cultura y herramientas; después la integración con tecnologías como nube y CI/CD; y finalmente las propuestas emergentes como lógica difusa, seguridad continua e híbridos ágiles \cite{LopezMenendezUnidad,ElAouni2025,Maidin2025,Aguayo2025,Ok2025}.

El objetivo principal es guiar la adopción de agilidad dentro de organizaciones que enfrentan los equipos, demandas cambiantes, riesgos de seguridad, múltiples plataformas tecnológicas, procesos distribuidos y necesidad de automatización. La metodología se construye como un marco práctico aplicable a proyectos de cualquier tamaño, combinando principios ágiles clásicos con prácticas contemporáneas, herramientas colaborativas y estrategias de integración tecnológica \cite{Khan2025,Ok2025,Shriram2025}.

Las investigaciones dan a entender que los equipos con mayor capacidad de maniobra responden mejor a cambios repentinos. Esta flexibilidad abarca la comunicación interna, la reorganización de prioridades y la actualización de herramientas \cite{LS2025,William2025}. 

La dinámica del equipo influye directamente en la calidad del producto. Factores como diversidad de género, comunicación transversal, claridad en los roles y revisión conjunta de tareas aumentan la efectividad general \cite{Stray2025}. 

La automatización, la nube, DevOps, las pruebas continuas, la documentación colaborativa, la lógica difusa y los tableros visuales se convierten en extensiones del trabajo humano, no en sustitutos. La tecnología amplifica el rendimiento del equipo cuando se integra con propósito \cite{ElAouni2025,Maidin2025,Aguayo2025,Solige2025,Shriram2025}. 

El proceso inicia cuando el equipo adopta una preparación cultural que permite que los cambios fluyan sin resistencia. En lugar de presentar la transformación como un ajuste drástico, se introduce de manera gradual, mostrando cómo ciertas prácticas pueden resolver problemas cotidianos que antes drenaban tiempo y energía, la cultura del equipo adquiere relevancia porque influye directamente en la forma en que las personas responden ante bloqueos, revisiones y ajustes. A medida que se construye esta base cultural, el equipo define roles de manera clara, aunque sin convertirlos en barreras que limiten la interacción, cada rol funciona como una pieza que aporta una perspectiva distinta. La presencia de perfiles diversos permite enriquecer las discusiones técnicas y equilibrar la distribución de tareas, lo que disminuye la posibilidad de que ciertos integrantes carguen con responsabilidades repetitivas o poco visibles. Se sugiere que el Product Owner participe activamente en la clarificación de requisitos y en la priorización del producto para evitar interpretaciones confusas, el Scrum Master se convierte en un facilitador que garantiza que las prácticas fluyan sin interrupciones innecesarias, mientras que el equipo técnico adopta una visión amplia del trabajo para no depender de especialistas aislados. La incorporación de un Security Champion agrega una capa de protección sin frenar el avance, pues permite que la seguridad se trate como parte del desarrollo y no como un control externo que aparece al final. Esta composición diversificada no pretende crear estructuras complejas; busca que cada integrante tenga claridad sobre su aporte y que la colaboración mantenga una dirección coherente \cite{Stray2025,LS2025,Ok2025}.

Con los roles establecidos, se avanza hacia la construcción de un ciclo de trabajo que se sostenga por sí mismo, el ciclo se organiza en iteraciones que permiten retroalimentación frecuente, reducen la incertidumbre y mantienen el producto en movimiento constante, el tablero visual adquiere un papel esencial porque traduce las decisiones del equipo en una representación clara y compartida. Cada columna refleja el estado real del trabajo y permite que cualquier integrante detecte irregularidades sin necesidad de solicitudes formales, esta transparencia genera responsabilidad colectiva, reduce tiempos muertos y facilita el ajuste del trabajo cuando cambian prioridades o surge un riesgo técnico imprevisto \cite{Moiseienko2025,Reich2025}. 

La automatización temprana forma parte del ciclo desde el primer día, por eso, plantea que el equipo desarrolle pruebas unitarias, análisis estático y validaciones integrales desde la primera iteración, este enfoque no solo mejora la calidad del código, sino que también acelera la detección de problemas. El pipeline de integración continua se convierte en una herramienta que acompaña cada decisión del equipo, su función no es únicamente ejecutar pruebas, sino ofrecer información constante sobre la estabilidad del producto, cuando un error aparece, se detecta en cuestión de segundos, lo que evita que el fallo avance a etapas donde sería más costoso corregirlo. Además, el equipo aprende a interpretar los resultados del pipeline como parte de su diálogo técnico, sin que se conviertan en simples números desconectados del proceso \cite{Solige2025,Shriram2025}. 

La seguridad integrada desempeña un papel similar, no como un obstáculo externo. Esto implica revisar dependencias, analizar vulnerabilidades y evaluar riesgos desde la misma planeación del sprint, gracias a esta integración, la calidad del producto aumenta de manera orgánica, ya que los problemas de seguridad se abordan antes de que escalen. Las organizaciones que aplicaron este enfoque, según las investigaciones revisadas, lograron reducir fallos críticos sin necesidad de controles adicionales que suelen frenar el avance del desarrollo. La clave está en que la seguridad se convierte en una práctica continua, accesible para todo el equipo, y no en un control impuesto \cite{Ok2025,Khan2025}. 

La arquitectura en la nube facilita que el desarrollo mantenga consistencia, la nube permite replicar entornos sin depender de configuraciones manuales y agiliza pruebas que requieren condiciones específicas. Esta capacidad de replicación mejora la fiabilidad de las pruebas y reduce discrepancias entre ambientes, lo que se traduce en menor cantidad de errores difíciles de reproducir, los despliegues se vuelven más estables y el equipo puede experimentar con mayor libertad sin comprometer el entorno principal. Este apoyo tecnológico se integra con DevOps, que agrega prácticas de despliegue continuo y mantenimiento de entornos estables \cite{ElAouni2025,Maidin2025}. 

La documentación ocupa un lugar fundamental en esta metodología porque las investigaciones muestran que los proyectos con documentación inconsistente tienden a acumular errores, la propuesta complementaria sugiere que la documentación se construya mediante trabajo en pares, esta estrategia aumenta la claridad, evita contradicciones y mantiene la información sincronizada con el trabajo real del equipo. La creación de diagramas y representaciones visuales facilita la comprensión de la lógica del sistema y permite que las decisiones técnicas se tomen con una visión más amplia del producto, esta documentación no se trata como un requisito formal, sino como una guía que permite que todos avancen con seguridad \cite{Qamar2025,JawishAlgorithm}.


