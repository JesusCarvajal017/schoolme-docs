\section{Estado del Arte Humanizado: Convergencia de Bienestar, Rigor y Ética}
El estado del arte revela que la relación entre la innovación tecnológica y el desarrollo humano ha sido abordada desde múltiples ángulos disciplinarios, todos los cuales convergen en la necesidad de equilibrar el potencial de la tecnología con el bienestar, el rigor y la ética. Esta revisión, fundamentada en diecisiete estudios, permite identificar patrones consistentes en torno a las vulnerabilidades humanas y metodológicas en el ecosistema digital.

I. El Factor Humano: Vulnerabilidades y Alfabetización Digital
La literatura especializada demuestra que la eficacia de los sistemas avanzados a menudo está limitada por el comportamiento y las capacidades biológicas del usuario.

Riesgo en Ciberseguridad y Comportamiento: En el campo de la ciberseguridad móvil, investigaciones como las de Prieto et al. \cite{Prieto2018} indican que el usuario suele subestimar la importancia de los permisos en sistemas operativos como Android. Este factor no es primordialmente técnico, sino un asunto de alfabetización digital y comportamiento, abriendo puertas a un creciente número de ataques y fugas de datos.

Ergonomía y Salud Visual (SVI): La exposición prolongada a pantallas ha generado problemas de salud ocupacional de creciente relevancia. Frometa et al. \cite{Frometa2012} y Miró \cite{Miro2005} evidencian que el Síndrome Visual Informático (SVI), junto a la fatiga cognitiva derivada de la privación de sueño, no solo afecta la visión, sino que disminuye el rendimiento cognitivo y la estabilidad emocional. Esto transforma la mala gestión tecnológica en un factor de deterioro silencioso con impacto directo en la productividad y la calidad de vida.

Inclusión y Adaptabilidad: En el diseño de software, la dimensión humana se extiende a la inclusión. Estudios como los de Pacheco et al. \cite{Pacheco2020} sobre software inclusivo para niños con Síndrome de Down leve demuestran que las decisiones de diseño tienen un impacto directo en las oportunidades de aprendizaje y la integración social, resaltando que el diseño ético debe ser fundamentalmente adaptativo.

II. Rigor Metodológico y Estandarización en Ingeniería
La complejidad inherente a los gigasistemas y a la multiplicidad de opciones tecnológicas exige que las decisiones de ingeniería se basen en métricas objetivas y rigurosas.

Calidad de Software (ISO/IEC 25000): La evaluación comparativa de frameworks web, como la realizada entre Laravel y Django por Espinosa \cite{Espinosa2021} y Tolosa \cite{Tolosa2014}, subraya que la calidad del software no debe basarse en modas o preferencias subjetivas, sino en criterios verificables y estandarizados conforme a normativas internacionales como la ISO/IEC 25000.

Interoperabilidad y Continuidad del Negocio: Bishop et al. \cite{Bishop2005} abordan el desafío de integrar sistemas heredados (como Java) con soluciones modernas (C\#). La interoperabilidad se presenta como una estrategia metodológica clave para la continuidad del negocio, asegurando que los sistemas puedan convivir sin requerir migraciones costosas y disruptivas.

Automatización como Necesidad: La creciente escala y complejidad de los sistemas requiere que la Automatización de Calidad (AQ) y el uso de Inteligencia Artificial para desarrolladores \cite{Rivas2025,Gruber2024} se conviertan en mecanismos de control y reducción de riesgo sistémico \cite{Simmons2024}, dado que la supervisión humana directa ya no es viable.

III. Tecnología, Ética y Mediación Pedagógica
Más allá de lo técnico, la literatura confirma que los sistemas avanzados plantean profundas disyuntivas éticas y no pueden sustituir la intervención pedagógica humana.

Riesgos de Vigilancia y Sesgo Algorítmico: Garvie \cite{Garvie2024} aborda el uso del reconocimiento facial, destacando que el avance de la automatización y la vigilancia podría escalar sin control si no existen marcos regulatorios claros y un diseño basado en la privacidad. Los riesgos éticos no son solo técnicos, sino sociopolíticos.

IA y el Rol Humano en el Desarrollo: La integración de la IA en el ciclo de desarrollo \cite{Rivas2025} cambia el rol del programador de ejecutor a arquitecto y validador. La IA gestiona la complejidad, pero el humano mantiene el juicio ético y la creatividad final.

Software Educativo y Mediación: Los estudios en el área de la didáctica, como los de Marquès \cite{MarquesSF} y Ribas-Xirgo \cite{Ribas2008}, indican que los sistemas de acompañamiento digital y los mapas conceptuales \cite{Alaminos2009} no sustituyen el rol del educador. Por el contrario, amplifican su capacidad para orientar, clarificar y evaluar la experiencia de aprendizaje, haciendo de la mediación pedagógica un elemento indispensable para el éxito tecnológico.

Conclusión de la Revisión: El análisis demuestra que, aunque los temas son diversos, existe una unidad conceptual irrefutable: la tecnología es un espejo de las decisiones humanas. Su valor real reside en su implementación consciente, ética y rigurosa.