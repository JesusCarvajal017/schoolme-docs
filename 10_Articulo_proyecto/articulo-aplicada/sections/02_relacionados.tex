\section{Marco teórico y trabajos relacionados}

Las investigaciones revisadas ofrecen una visión amplia de cómo las metodologías ágiles se han consolidado, transformado y combinado con otros enfoques. Aunque cada artículo aborda un aspecto distinto, todos coinciden en que Agile no es solamente una técnica de trabajo: es una cultura organizacional basada en adaptabilidad, colaboración y mejora continua. Esta sección integra los principales aportes de los trabajos revisados, mostrando cómo se conectan entre sí y cómo dan forma al desarrollo actual.

Uno de los puntos claves del estado del arte es la consolidación histórica de Agile. López Menéndez de Jiménez describe cómo desde 2001 las organizaciones comenzaron a priorizar la adaptabilidad por encima de la planificación rígida y cómo metodologías como Scrum se volvieron populares gracias a su estructura de roles, ciclos cortos
retroalimentación frecuente \cite{LopezMenendezUnidad}. Este trabajo marca el punto de partida para entender por qué Agile empezó a desplazar enfoques tradicionales.

Otro eje relevante es el análisis de Agile en países en desarrollo, como lo muestran Shafir y colaboradores en Bangladesh. Señalan que, aunque Agile promete mejoras en tiempos y calidad, los entornos con limitaciones de experiencia, capacitación o recursos enfrentan retos especiales. Este tipo de estudios demuestra que la implementación exitosa depende del contexto y no solo de seguir buenas prácticas \cite{LS2025}.

La diversidad dentro de los equipos también aparece como un factor importante. Stray et al. analizan cómo los grupos con mayor equilibrio de género tienden a tener mejores dinámicas de colaboración y distribución de tareas. No es un tema técnico, sino humano, pero afecta directamente al rendimiento y la calidad del trabajo \cite{Stray2025}. Esto recuerda que las metodologías no operan solas, dependen de las personas que las aplican.

Varias investigaciones comparan o integran metodologías. Jain estudia diferencias en el desarrollo de iOS y Android cuando se aplica Scrum o Kanban, mostrando que cada entorno tiene particularidades que influyen en la elección de prácticas \cite{Jain2025}. De manera similar, Moiseienko et al. presentan una aplicación Kanban y explican cómo cada herramienta o metodología debe ajustarse al ritmo del equipo \cite{Moiseienko2025}.

La globalización del desarrollo tiene importancia a la vez. Reich y Reich analizan el uso de Scrum en equipos distribuidos y señalan desafíos como diferencias culturales, zonas horarias y barreras de comunicación. Destacan que la agilidad exige liderazgo empático, no solo reglas \cite{Reich2025}.

Algunas investigaciones profundizan en la integración de Agile con modelos modernos como DevOps y la nube. El Aouni y colegas muestran cómo Agile, Cloud y DevOps se complementan para ofrecer escalabilidad, automatización y velocidad \cite{ElAouni2025}. Maidin et al. examinan tendencias futuras y explican por qué la seguridad debe tener el mismo nivel de importancia que la funcionalidad \cite{Maidin2025}.

También aparecen enfoques híbridos que combinan agilidad con estructuras tradicionales, como Agile Híbrido o la integración de Agile con Lean Six Sigma, propuesta por Jibgah y colaboradores \cite{Jibgah2025}. Este tipo de aproximaciones muestra que las organizaciones requerir flexibilidad sin perder control sobre la calidad y los tiempos.

La calidad del software y la documentación también reciben atención. Qamar et al. evalúan la documentación en pares y encuentran mejoras significativas en la claridad y consistencia de los requisitos \cite{Qamar2025}. Por su parte, Malla compara la velocidad y calidad entre metodologías ágiles y tradicionales, mostrando que Agile reduce tiempos sin sacrificar estabilidad \cite{PriyankaMalla2025}.

Solige analiza las pruebas automatizadas \cite{Solige2025}, mientras Cubillos presenta un caso práctico en un entorno e-commerce aplicando Scrum y Selenium IDE \cite{CubillosGarcia2025}. Ambos trabajos muestran cómo la automatización se vuelve un aliado para mantener calidad en procesos rápidos.

Varios artículos abordan medición y optimización. Aguayo et al. utilizan lógica difusa para medir agilidad de manera más precisa \cite{Aguayo2025}, mientras Shriram explora estrategias para mejorar el flujo CI/CD \cite{Shriram2025}, mostrando que las métricas importan tanto como las prácticas de trabajo. Estos trabajos muestran que Agile no es un enfoque aislado. Se conecta con seguridad, pruebas, automatización, documentación, nube, cultura organizacional, diversidad, estadística, liderazgo y modelos híbridos.

Las distintas investigaciones analizadas coinciden en que la agilidad ha dejado de ser un método de desarrollo para consolidarse como una cultura organizacional basada en la adaptabilidad, la colaboración y el aprendizaje continuo. Desde los primeros estudios que explicaron los fundamentos del Manifiesto Ágil y su aplicación en entornos empresariales (López Menéndez de Jiménez, s. f.) \cite{LopezMenendezUnidad}, se evidenció que la clave del éxito no radica únicamente en las herramientas, sino en la capacidad de los equipos para adaptarse a los cambios y generar valor de manera constante.

Investigaciones en contextos de países en desarrollo, como la de Shafir et al. (2025) \cite{LS2025}, mostraron que los factores de éxito en la implementación de metodologías ágiles dependen tanto de la planificación y la comunicación como del compromiso con la mejora continua. Estas conclusiones se complementan con los hallazgos de Stray et al. (2025) \cite{Stray2025}, quienes demostraron que la diversidad de género dentro de los equipos ágiles influye positivamente en la calidad del trabajo y en la distribución de tareas, fortaleciendo las dinámicas colaborativas.

Por otra parte, Jain (2025) \cite{Jain2025} exploró las diferencias en la aplicación de Scrum y Kanban en entornos de desarrollo para iOS y Android, resaltando que la agilidad no es una receta uniforme, sino una mentalidad que debe adaptarse a las características y necesidades de cada ecosistema. En una línea similar, Reich y Reich (2025) \cite{Reich2025} destacaron que el uso de Scrum en proyectos globales exige no solo coordinación técnica, sino liderazgo, empatía y comunicación para mantener la cohesión entre equipos distribuidos.

Los estudios de Moiseienko, Moiseienko y Lubentsova (2025) \cite{Moiseienko2025} compararon la efectividad de Scrum y Kanban, concluyendo que la agilidad no depende de las herramientas, sino del aprendizaje y la mejora incremental del equipo. De manera complementaria, William (2025) \cite{William2025} evaluó la eficacia de las metodologías de desarrollo en ecosistemas modernos, confirmando que la agilidad, frente a enfoques tradicionales como Waterfall, ha ganado predominio por su capacidad de adaptación y colaboración.

Desde una perspectiva gerencial, LS et al. (2025) \cite{LS2025} identificaron los principales desafíos en la implementación de metodologías ágiles, como la resistencia al cambio y la falta de experiencia, y propusieron estrategias basadas en liderazgo, capacitación y comunicación. En paralelo, El Aouni et al. (2025) \cite{ElAouni2025} revisaron la integración de Agile, Cloud y DevOps, señalando que la sinergia entre estas prácticas permite optimizar los recursos y alcanzar procesos más escalables y eficientes.

Otros estudios profundizan en dimensiones más específicas. Khan y Ali (2025) \cite{Khan2025} examinaron los retos de incorporar DevOps dentro del marco del Project Management Institute (PMI), resaltando la importancia de combinar la disciplina de la gestión tradicional con la velocidad y automatización propias de la agilidad. Malla (2025) \cite{PriyankaMalla2025} confirmó, a través de un estudio comparativo, que el uso de Agile mejora tanto la calidad del software como la velocidad de entrega, especialmente al integrarse con inteligencia artificial y estrategias Cloud-native.

En cuanto a la seguridad, Ok y Eniola (2025) \cite{Ok2025} abordaron las mejores prácticas para la gestión de riesgos en entornos ágiles, subrayando que la ciberseguridad debe incorporarse desde las etapas iniciales mediante enfoques DevSecOps. De forma paralela, Maidin et al. (2025) \cite{Maidin2025} y Qamar et al. (2025) \cite{Qamar2025} destacaron la relevancia de integrar seguridad y documentación colaborativa para mejorar la sostenibilidad y la calidad de los requisitos en proyectos ágiles.

También se refleja en propuestas como la de Aguayo et al. (2025) \cite{Aguayo2025}, quienes emplearon lógica difusa para medir la agilidad mediante modelos de inteligencia artificial, y en la de Cubillos García (2025) \cite{CubillosGarcia2025}, que aplicó Scrum y pruebas automatizadas E2E en entornos de comercio electrónico. Jawish et al. (2025) \cite{JawishAlgorithm} propusieron el enfoque Algorithm-Driven Development (ADD), que prioriza la validación visual y estructurada del software desde su origen.

Por último, investigaciones como las de Solige (2025) \cite{Solige2025} y Shriram (2025) \cite{Shriram2025} demuestran que la automatización de pruebas y la optimización de flujos CI/CD son pilares fundamentales para alcanzar un equilibrio entre velocidad, calidad y estabilidad.

