\section{Resultados: La Consolidación de los Tres Ejes Transversales}
El análisis comparativo y la síntesis interpretativa de los diecisiete estudios revisados revelaron de manera consistente y recurrente tres hallazgos clave. Estos resultados trascienden las áreas disciplinares individuales (ingeniería, salud, educación) y se establecen como principios rectores para el diseño y la implementación tecnológica en la era digital:

1. La Tecnología Amplifica el Valor Humano, No lo Sustituye
Se encontró que, incluso en los sistemas más avanzados y automatizados, la calidad final y el valor estratégico dependen irrevocablemente del factor humano. La tecnología funciona como un catalizador o herramienta de aumento cognitivo (cognitive augmentation), no como un reemplazo integral.

IA y Productividad: La integración de la Inteligencia Artificial al ciclo de desarrollo acelera la productividad del programador al automatizar tareas repetitivas \cite{Rivas2025}. Sin embargo, la arquitectura, el juicio ético y la creatividad final siguen siendo competencias exclusivamente humanas \cite{Gruber2024}. Un error o un sesgo en la IA son, en última instancia, fallos en el criterio o la supervisión del equipo humano.

Mediación Pedagógica: Los softwares educativos solo son efectivos si están acompañados de una intervención pedagógica humana cualificada \cite{MarquesSF, Ribas2008}. El resultado en el aprendizaje depende de la capacidad del educador para guiar, evaluar y adaptar la herramienta digital.

Seguridad y Comportamiento: La ciberseguridad móvil \cite{Prieto2018} sigue demostrando que la vulnerabilidad más crítica reside en el comportamiento del usuario y su falta de formación, lo que consolida al factor humano como el eslabón de máxima criticidad.

2. La Complejidad Digital Exige Rigor Metodológico y Estandarización
La escala y la interconexión de los gigasistemas modernos han superado la capacidad de gestión manual. Por lo tanto, el rigor metodológico y la estandarización se vuelven indispensables para la supervivencia y la calidad sistémica, más allá de ser una mera formalidad.

Métricas Objetivas: La evaluación de la calidad de frameworks o software sin el uso de métricas objetivas y estandarizadas (como las especificadas en ISO/IEC 25000), conduce inevitablemente a decisiones ineficientes o sesgadas por preferencias personales \cite{Espinosa2021}.

Automatización de Calidad (AQ): La automatización en gigasistemas \cite{Simmons2024} se establece como la única estrategia viable para garantizar la calidad y la trazabilidad. La complejidad exige la regulación algorítmica de los procesos para evitar errores sistémicos que superan la capacidad humana de detección.

Trazabilidad en Investigación: Incluso en el ámbito cualitativo, el uso de herramientas como QDAS \cite{Costa2016} permite una trazabilidad y un rigor que elevan la validez de los estudios, demostrando que la sistematización es un requisito en todas las disciplinas.

3. Toda Innovación Implica Riesgos que Deben ser Gestionados Proactivamente
La introducción de nuevas tecnologías siempre conlleva un conjunto de riesgos emergentes (técnicos, éticos y de salud) que deben ser anticipados e integrados en la fase de diseño del proyecto.

Riesgos Éticos y Legales: El desarrollo de sistemas como el reconocimiento facial \cite{Garvie2024} implica riesgos de vigilancia masiva y erosión de la privacidad. La gestión de riesgos no puede limitarse a bugs o exploits, sino que debe abordar las implicaciones sociales y legales del uso de datos.

Riesgos de Salud y Bienestar: El uso intensivo de pantallas genera riesgos de salud directa, como el deterioro visual (SVI \cite{Frometa2012}) y la fatiga cognitiva derivada de la privación de sueño \cite{Miro2005}. Estos no son fallos del usuario, sino riesgos operativos del sistema digital que deben mitigarse con diseño ergonómico y límites de uso.

Amenazas Integradas: Los riesgos éticos, de salud y técnicos convergen. Un sistema con sesgos algorítmicos (riesgo ético) genera decisiones injustas, lo que se traduce en riesgo operacional y, finalmente, riesgo legal.

Estos resultados consolidan la tesis central del artículo: es imperativo integrar siempre los factores humanos, metodológicos y éticos en cualquier desarrollo o intervención digital para alcanzar una calidad integral.