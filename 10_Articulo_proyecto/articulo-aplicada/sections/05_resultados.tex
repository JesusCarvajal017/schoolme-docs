\section{Evaluación y resultados}
El análisis de las investigaciones revisadas permite comprender que la agilidad no es únicamente una metodología para gestionar proyectos, sino un marco de pensamiento que transforma la cultura organizacional y la manera en que las personas perciben el trabajo. La discusión central gira en torno a la capacidad de las metodologías ágiles para integrar tres dimensiones que antes se trataban por separado: la eficiencia técnica, la colaboración humana y la adaptabilidad al cambio \cite{William2025,PriyankaMalla2025}.

Scrum y Kanban continúan siendo los pilares de la práctica ágil debido a su simplicidad y efectividad en la organización del trabajo. Sin embargo, su éxito depende menos de su estructura formal que del compromiso y la comunicación entre los miembros. Esta observación se refuerza en los estudios de Reich y Reich (2025), quienes demostraron que la distancia física o la diversidad cultural no constituyen un obstáculo cuando existe liderazgo empático y cohesión en los equipos \cite{Reich2025,Moiseienko2025}.

Los hallazgos de Shafir et al. (2025) y Stray et al. (2025) amplían la comprensión del enfoque ágil al mostrar que su éxito no se limita a contextos tecnológicamente avanzados. La experiencia de países en desarrollo y los análisis sobre diversidad de género confirman que la adaptabilidad y la inclusión son factores determinantes para el rendimiento y la innovación. Estas perspectivas demuestran que la agilidad es también una herramienta de democratización dentro del entorno empresarial, capaz de equilibrar las diferencias y potenciar las capacidades individuales \cite{LS2025,Stray2025}.

Otro punto clave es la evolución del pensamiento ágil hacia una integración con tecnologías emergentes, la fusión con DevOps, Cloud y Lean Six Sigma representa una tendencia hacia modelos híbridos más completos, donde la automatización y el análisis de datos fortalecen la calidad y la sostenibilidad de los procesos. Estos avances técnicos se acompañan de una transformación en la mentalidad gerencial. En las investigaciones de LS et al. (2025) se observa que el liderazgo ágil no se basa en el control, sino en la guía, la escucha y la confianza. Así, el papel del gestor tiene un nuevo significado, ya no es quien dicta el rumbo del proyecto, sino quien facilita el entorno para que el equipo pueda autogestionarse y aprender de manera continua \cite{ElAouni2025,Jibgah2025,Maidin2025,LS2025,Khan2025}.

De igual forma, los trabajos centrados en seguridad y en la optimización de flujos CI/CD destacan que la velocidad y la calidad no deben ser metas opuestas, el verdadero valor de la agilidad se encuentra en el equilibrio entre ambos factores, donde la automatización, la prevención de errores y la mejora constante garantizan productos más confiables y sostenibles \cite{Solige2025,Shriram2025,Ok2025}. 

Se entiende que en los resultados discutidos las metodologías ágiles han madurado desde un enfoque centrado en el software hacia un modelo de pensamiento que puede aplicarse a diversos sectores, su eficacia depende de la capacidad de las organizaciones para asumir el cambio como una oportunidad y no como una amenaza. La flexibilidad, la comunicación y el aprendizaje colectivo son los motores que sostienen esta creencia \cite{William2025,PriyankaMalla2025}.

Como conclusión general, puede decirse que las metodologías ágiles no solo han mejorado la calidad y rapidez del desarrollo de software, sino que han dado una nueva definición a la naturaleza del trabajo en equipo y la gestión del conocimiento, su integración con metodologías como Lean Six Sigma, DevOps o PMI muestra que la agilidad puede convivir con estructuras tradicionales sin perder su esencia adaptativa. La incorporación de la inteligencia artificial y la automatización anticipa una nueva etapa en la que la agilidad será también medible, predecible y más cercana al usuario final \cite{Jibgah2025,Khan2025,ElAouni2025,JawishAlgorithm,Aguayo2025}.

La aplicación del marco integrado produjo efectos visibles que no sustituyen los resultados previos de la organización, sino que los amplían desde una perspectiva práctica y tecnológica. A medida que el equipo puso en marcha los ciclos iterativos, la automatización temprana y la seguridad incorporada desde el inicio, comenzaron a aparecer cambios que no dependieron de ajustes formales, sino del modo en que las personas interactuaron con el flujo de trabajo. La observación del comportamiento diario del equipo mostró mejoras que surgieron casi de manera espontánea, impulsadas por la transparencia del tablero, el uso constante de pipelines estables y la claridad que trajeron las revisiones colaborativas \cite{Solige2025,Shriram2025,Ok2025,Maidin2025,Qamar2025}.

Uno de los cambios más evidentes fue la forma en que el equipo abordó los errores, la presencia de pruebas automáticas permitió detectar fallos en cuestión de segundos, lo que redujo el tiempo que antes se invertía en buscar causas ocultas o reproducir escenarios no tan claros, la rapidez en la detección cambió la percepción del error, dejó de verse como una amenaza para el avance del sprint y pasó a convertirse en un recordatorio de que el flujo funcionaba como debía. La reducción de incertidumbre también disminuyó la presión durante los cierres de ciclo, ya que el equipo sabía que la mayor parte de los problemas técnicos se revelaban antes de llegar a la fase de revisión \cite{Solige2025,Shriram2025}.

\begin{figure}[h!]
    \centering
    \includegraphics[width=0.90\linewidth]{graphics/grafica6.png}
    \caption{Resultados del proceso de automatización temprana y detección continua de errores mediante CI/CD.}
    \label{fig:grafica6}
\end{figure}


La integración de seguridad dentro del sprint tuvo un efecto similar, la revisión constante del código, el análisis de dependencias y la validación de amenazas evitaron que los problemas se acumularan, el equipo empezó a anticiparse a escenarios que antes no consideraban, lo que contribuyó a que la calidad del producto aumentara sin necesidad de incorporar controles adicionales, la seguridad dejó de ser una carga para convertirse en una parte natural del proceso y esa transición redujo significativamente la aparición de vulnerabilidades \cite{Ok2025,Khan2025,Maidin2025}.

El trabajo en equipo se fortaleció cuando el intercambio de tareas, la rotación de responsabilidades y las decisiones técnicas revisadas en pares mejoraron la comprensión colectiva del sistema. La dinámica diaria cambió y las conversaciones se volvieron más fluidas, los bloqueos se resolvían sin fricción y las discusiones técnicas adquirieron un enfoque más práctico. Los integrantes se apoyaron mutuamente para superar dificultades técnicas y evitar acumulación de trabajo en manos de un solo perfil, esta distribución equilibrada no solo aumentó la eficiencia, sino que generó un ambiente de confianza donde cada persona podía intervenir sin temor a cometer errores \cite{Stray2025,LS2025}.

Los tableros visuales contribuyeron a un cambio en la percepción del avance, la visibilidad constante del flujo permitió que los integrantes anticiparan retrasos antes de que afectaran al sprint completo, cuando un elemento permanecía inmóvil más tiempo del esperado, la reacción del equipo surgía sin necesidad de que alguien solicitara intervenir. Esta capacidad de detectar bloqueos temprano redujo interrupciones, mejoró la continuidad del trabajo y mantuvo un ritmo estable a lo largo de los ciclos. El proyecto dejó de depender de reportes tardíos o reuniones extensas para identificar cuellos de botella \cite{Moiseienko2025,Reich2025}.

El uso de entornos en la nube generó otro conjunto de resultados. La eliminación de diferencias entre ambientes disminuyó fallos derivados de configuraciones inconsistentes. Las pruebas de despliegue se realizaron con mayor confianza y los cambios pudieron ejecutarse sin generar impactos inesperados. La replicación rápida de entornos facilitó experimentos controlados, lo que ayudó a afinar decisiones importantes sin arriesgar la estabilidad del sistema principal. Las tareas que antes requerían coordinación entre múltiples áreas se resolvieron con mayor rapidez, ya que el equipo tenía acceso directo a las herramientas necesarias \cite{ElAouni2025,Maidin2025}.

El trabajo colaborativo sobre la documentación contribuyó a consolidar una base de conocimiento clara, los documentos reflejaron mejor las decisiones reales del proyecto y permitieron que nuevos participantes comprendieran la estructura del sistema con rapidez. Las revisiones en pares evitaron contradicciones y ofrecieron un estilo más uniforme, lo que eliminó interpretaciones ambiguas durante el desarrollo. Esta mejora en la claridad documental influyó directamente en la calidad de las funcionalidades, pues el equipo redujo la cantidad de ajustes derivados de malentendidos \cite{Qamar2025,JawishAlgorithm}.

El monitoreo constante del pipeline, el análisis del tiempo de ciclo y la revisión del flujo permitieron detectar patrones que antes pasaban desapercibidos. El equipo aprendió a tomar decisiones basadas en datos y no en percepciones fragmentadas. Esta nueva forma de interpretar los avances favoreció ajustes más precisos, pues los integrantes podían identificar con claridad qué parte del proceso requería refuerzo y cómo debía abordarse esa necesidad. El análisis mediante lógica difusa ofreció una visión más matizada de la madurez del equipo, mostrando no solo el progreso, sino también los matices entre fortalezas y áreas de mejora \cite{Aguayo2025,Shriram2025,Solige2025}.

\begin{figure}[h!]
    \centering
    \includegraphics[width=0.90\linewidth]{graphics/grafica7.png}
    \caption{Evaluación del rendimiento del equipo mediante métricas de flujo y análisis basado en lógica difusa.}
    \label{fig:grafica7}
\end{figure}


La capacidad para responder a cambios repentinos ganó solidez. Cuando surgieron variaciones en prioridades o el Product Owner ajustó la dirección del producto, el equipo pudo reorganizar el flujo sin generar desorden. La presencia de automatización, documentación clara y entornos replicables permitió implementar modificaciones sin comprometer la estabilidad del sprint. La organización observó entregas estables incluso en periodos de alta demanda, lo que incrementó la confianza en el equipo y facilitó la toma de decisiones estratégicas \cite{William2025,Reich2025,LS2025}.

La experiencia acumulada a lo largo de varios ciclos también generó una mejora sostenida en el ritmo de entrega. Aunque el objetivo no era aumentar velocidad sin control, la reducción de errores, la estabilidad del pipeline, la claridad en los roles y la seguridad integrada implicaron que el equipo entregara funcionalidades con mayor regularidad. El tiempo invertido en corrección disminuyó de manera natural, lo que permitió dedicar más esfuerzo a tareas de valor real. La predictibilidad del sprint creció, y esa estabilidad benefició tanto al equipo como a la dirección \cite{PriyankaMalla2025,Solige2025,Shriram2025,Ok2025}.
