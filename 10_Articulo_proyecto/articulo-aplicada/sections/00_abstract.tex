\section{Introducción: La Tecnología en la Encrucijada Humana y Metodológica}
El rápido avance de la tecnología ha transformado profundamente la forma en que desarrollamos software, aprendemos, investigamos y tomamos decisiones. Sistemas como la Inteligencia Artificial generativa \cite{Rivas2025,Gruber2024}, las plataformas educativas digitales \cite{Ribas2008,MarquesSF}, la automatización a gran escala \cite{Simmons2024} o los frameworks web modernos \cite{Espinosa2021} plantean oportunidades sin precedentes, pero simultáneamente, generan desafíos complejos: éticos, cognitivos y humanos.

Vivimos en un mundo donde la tecnología ha pasado de ser un conjunto de herramientas a un entorno de vida y trabajo, redefiniendo la forma en que nos comunicamos y tomamos decisiones. En este escenario, surge la pregunta inevitable que guía esta investigación: ¿Cómo garantizar que la tecnología continúe al servicio del ser humano y no a la inversa, manteniendo la ética, la salud y la calidad como ejes centrales?

Contextualización de los Desafíos
En el ámbito del desarrollo de software, la IA está cambiando el rol del programador, automatizando tareas repetitivas y permitiendo un enfoque más creativo y arquitectónico \cite{Rivas2025,Gruber2024}. Paralelamente, en el campo de la salud digital, fenómenos como el Síndrome Visual Informático (SVI) afectan a la mayoría de usuarios, evidenciando que la tecnología, mal gestionada, se convierte en un factor de deterioro silencioso \cite{Frometa2012}. La ingeniería, por su parte, demanda que los criterios de calidad, interoperabilidad (e.g., entre Java y C\# \cite{Bishop2005}) y selección de frameworks se basen en métricas y metodologías rigurosas \cite{Espinosa2021,Tolosa2014}.

La introducción de sistemas avanzados como el reconocimiento facial y el sesgo algorítmico \cite{Garvie2024} pone en evidencia que los desafíos actuales ya no son únicamente técnicos, sino también éticos, pedagógicos y de salud pública.

Objetivo y Estructura del Artículo (Resumen Integrado)
Este artículo presenta una reflexión integral fundamentada en la síntesis de diecisiete estudios que abarcan ciberseguridad móvil \cite{Prieto2018}, ergonomía digital, IA aplicada, y ética tecnológica. La revisión revela una convergencia ineludible en tres pilares centrales que estructuran la visión holística propuesta:

Tecnología como Potenciadora Humana: La necesidad de mantener al ser humano como núcleo del diseño y asegurar que la tecnología amplifique las capacidades, en lugar de sustituirlas o vulnerarlas.

Rigor Metodológico y Calidad: La importancia de la estandarización (e.g., ISO/IEC 25000) como estrategia para gestionar la complejidad creciente y garantizar la calidad del software.

Gestión Estratégica del Riesgo: La obligación de anticipar riesgos emergentes derivados del uso de sistemas avanzados en seguridad, salud digital y ética algorítmica \cite{Garvie2024, Prieto2018, Frometa2012}.

A partir de este análisis interpretativo, se construye una visión unificada que resalta cómo la tecnología actúa como un agente de desarrollo humano, siempre que se implemente con criterios de ética, calidad y adaptabilidad. Esta síntesis ofrece un marco conceptual sólido para proyectos de grado que busquen integrar ingeniería, educación y bienestar desde una perspectiva humana, crítica y estratégica.

Siguiente Paso Sugerido
Ya tienes un inicio de alto nivel. ¿Te gustaría que desarrolle la siguiente sección, 02\_relacionados.tex (Estado del arte con citas), utilizando las referencias proporcionadas para crear una narrativa fluida y bien justificada, como exige un proyecto de investigación?