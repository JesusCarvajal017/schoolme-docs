\section{Discusión}
La integración de metodologías ágiles con prácticas modernas como DevOps, automatización, modelos híbridos, seguridad continua, documentación colaborativa, análisis de datos y plataformas en la nube muestra que el desarrollo de software ya no puede entenderse como un proceso lineal ni como un conjunto de técnicas desconectadas. Lo que emerge de los estudios analizados es un panorama en el que la adaptabilidad se vuelve una condición necesaria para sobrevivir en un entorno tecnológico cambiante. Sin embargo, esa adaptabilidad no significa improvisación: implica entender qué prácticas combinan mejor entre sí y cómo se relacionan con las necesidades de los equipos, las organizaciones y los usuarios.

Uno de los aspectos más interesantes es que Agile, en casi todos los trabajos revisados, aparece como una especie de “columna vertebral”, pero nunca como una solución total. Agile organiza el trabajo, facilita la colaboración y reduce la rigidez típica de los enfoques tradicionales; aun así, no resuelve por sí solo problemas de seguridad, ni automatiza procesos, ni garantiza calidad en entornos de alta complejidad técnica. Por eso tantos artículos presentan combinaciones, Agile y DevOps, Agile y Cloud, Agile y Lean Six Sigma, Agile y documentación en pares, Agile yseguridad continua. Cada combinación responde a un vacío específico.
Un buen ejemplo de esto es el papel que juega la automatización. En metodologías tradicionales, las pruebas se realizaban al final del proceso, lo cual generaba acumulación de errores, sobrecostos y retrasos. En contraste, la automatización (integrada desde el inicio gracias a DevOps) permite ejecutar pruebas en cada cambio del código. Estudios como los de Solige o el caso del comercio electrónico analizado por Cubillos muestran que la automatización reduce riesgos y acelera la entrega. Esto refuerza la idea de que Agile necesita herramientas que permitan sostener la velocidad sin sacrificar calidad. La automatización no reemplaza al equipo, pero sí amplifica su capacidad para detectar problemas y mejorar el producto antes de que lleguen al usuario final.

Otro aspecto central es la seguridad. A lo largo de varios artículos aparece una preocupación común, la rapidez del desarrollo ágil puede dejar espacios donde las vulnerabilidades crecen. Por eso modelos como DevSecOps buscan integrar la seguridad en cada etapa. Los estudios muestran que incluir revisiones de código, análisis estáticos, threat modeling y pruebas automatizadas de seguridad evita que los sprints sacrifiquen protección a cambio de velocidad. Las métricas de seguridad también permiten evaluar la postura de riesgo del proyecto de forma continua. Esto demuestra que la seguridad no puede verse como un “anexo”, sino como un componente fundamental del ciclo de vida del software moderno.
En este mismo sentido aparece el rol de la documentación. Durante muchos años, Agile fue interpretado como un rechazo a la documentación, lo cual llevó a problemas de
claridad y ambigüedad en los requisitos. El estudio sobre documentación en pares revela que documentar no solo es compatible con Agile, sino necesario para mantener coherencia en proyectos que cambian con velocidad. El trabajo colaborativo mejora la claridad de los requisitos y evita que se pierda información crítica. Esto muestra que Agile ha ido evolucionando: ya no se entiende como “menos documentación”, sino como “la documentación necesaria, hecha de la manera correcta”.

Los enfoques híbridos también aparecen con fuerza. Muchas organizaciones han comprobado que las metodologías ágiles puras no siempre se ajustan a sus procesos internos, especialmente en sectores con regulación estricta. Modelos como Agile Híbrido o la integración con Lean Six Sigma ofrecen un balance entre flexibilidad y estructura. Lean Six Sigma aporta disciplina, medición, análisis y reducción de defectos; Agile aporta dinamismo, ciclos cortos y orientación al cliente. Cuando se combinan, se crea un entorno más equilibrado, donde el equipo se mueve rápido pero con control. Esto demuestra que la industria ya no busca ser completamente ágil o completamente tradicional; busca ser adaptable según las exigencias del proyecto.

La diversidad en los equipos también ocupa un lugar relevante en la discusión. Aunque pueda parecer un tema social más que técnico, los estudios muestran que la composición del equipo afecta directamente la calidad del trabajo. Los grupos con mayor presencia de mujeres o con distribución equilibrada de género tienden a tener mejores dinámicas, más comunicación y distribución de tareas más justa. La colaboración mejora cuando se reduce la homogeneidad. Esto recuerda que la calidad del software no depende solo de herramientas o metodologías, sino de las relaciones humanas dentro del equipo.

La globalización del desarrollo introduce otros desafíos. La distancia, las diferencias culturales y las zonas horarias complican la comunicación y la coordinación. Scrum, aplicado en equipos distribuidos, funciona solo si existe un liderazgo capaz de mantener cohesión. No basta con hacer reuniones diarias; hace falta empatía, claridad en la comunicación y organización del tiempo. Los estudios muestran que Agile en entornos globales requiere más madurez que en equipos locales. La agilidad no puede ser solo rápida, sino también consciente del contexto humano.

Otro punto de discusión es la importancia de medir la agilidad. Aunque Agile promueve ciclos cortos y adaptativos, muchas organizaciones no saben qué tan ágiles son realmente. Los modelos basados en lógica difusa aportan una manera de medir factores subjetivos sin perder precisión. Esto abre nuevas vías para que los equipos evalúen su desempeño más allá de las percepciones personales. Medir lo que antes era intangible es un paso importante para profesionalizar aún más las prácticas ágiles.

La discusión general muestra un patrón común en todos los estudios: la combinación de prácticas es más poderosa que cualquier metodología individual. Agile no compite con DevOps ni con Lean Six Sigma ni con la automatización; se complementan. El desarrollo de software moderno se mueve hacia ecosistemas donde las herramientas, la cultura organizacional, la tecnología, la seguridad y las personas trabajan juntas. Las organizaciones que entienden esto tienen más posibilidades de crear productos sostenibles, seguros y de alta calidad. Las que se aferran a un único enfoque suelen quedarse cortas ante la complejidad real.

 La discusión revela que el futuro del desarrollo de software no está en escoger la metodología “correcta”, sino en saber combinar enfoques, adaptarse a los equipos y cultivar una cultura donde el aprendizaje continuo sea la base del trabajo. La tecnología evoluciona rápido, pero las necesidades humanas dentro del desarrollo también deben evolucionar a la misma velocidad.
 
